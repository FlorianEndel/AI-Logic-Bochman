%%% Choose between 16:9 and 4:3 format by commenting out/uncommenting one of the following lines:
%\documentclass[aspectratio=169]{beamer} % 16:9
\documentclass{beamer} % 4:3



%=========================================================================================================================

\usepackage{beamerthemesplit}
\usepackage{appendixnumberbeamer} % do not count appendix for total page number
\usepackage[english]{babel}
%\usepackage[utf8]{inputenc}

\usepackage{hyperref}
\usepackage{xurl}


\usepackage{tikz}
\usepackage{url}
\usepackage{tabularx}

\useinnertheme{aig}
\useoutertheme{aig}
\usecolortheme{aig}
\usefonttheme{aig}

%% fonts
\usepackage{fontspec}
\setmainfont{Crimson}


%%% notes
%\setbeameroption{hide notes} % Only slides
%\setbeameroption{show only notes} % Only notes
\setbeameroption{show notes on second screen=right} % Both


%% Table of Contents      
\AtBeginSection[]{
	\begingroup 
		%\setbeamertemplate{headline}{} 
		%\setbeamertemplate{footline}{} 
		\addtobeamertemplate{frametitle}{\vspace*{0\baselineskip}}{}
		
		\begin{frame}<beamer> 
			\frametitle{Table of Contents} 
			\tableofcontents[currentsection,currentsubsection] 
			%\addtocounter{framenumber}{-1}
		\end{frame}
	\endgroup
} 
 
%% Table of Content: Repeat at beginning of subsection      
% \AtBeginSubsection[]{
% 	\begingroup 
% 		\setbeamertemplate{headline}{} 
% 		\setbeamertemplate{footline}{} 
% 		\addtobeamertemplate{frametitle}{\vspace*{0\baselineskip}}{}
% 		
% 		\begin{frame}<beamer> 
% 			\frametitle{Inhalt} 
% 			\tableofcontents[currentsection,currentsubsection] 
% 			\addtocounter{framenumber}{-1}
% 		\end{frame}
% 	\endgroup
% } 
 


%% References
\usepackage{csquotes}
\usepackage[
    backend=biber,
    style=ieee, %
    natbib=true,
    style=numeric,
    sorting=none,
    url=false,
    citecounter=true,
    citetracker=true,
    maxcitenames=1
] {biblatex} % authoryear-icomp
\addbibresource{references.bib}

% Font size in list of references 
\renewcommand*{\bibfont}{\scriptsize}


%% citation in footnotes
\let\svthefootnote\thefootnote

% https://tex.stackexchange.com/a/29931/38244
\DeclareCiteCommand{\footcite}
{\usebibmacro{prenote}}%
{%  
    \ifciteseen{}{%
        \usebibmacro{citeindex}%
        \let\thefootnote\relax%
        \footnotetext{%
            \mkbibbrackets{\usebibmacro{cite}}%
            \setunit{\addnbspace}
            \printnames{labelname}%
            \setunit{\labelnamepunct}
            \printfield[citetitle]{title}%
            \newunit%
            \printfield[]{year}%
        }%
        \let\thefootnote\svthefootnote%
    }%
    \autocite{\thefield{entrykey}}%
  }
{\addsemicolon\space}
%{\usebibmacro{postnote}}

%% customcite
\newcommand{\customcite}[1]{\cite{#1} \citeauthor{#1}. \citetitle{#1}. \citeyear{#1}}
\newcommand{\customciteNoAuthor}[1]{\cite{#1} \citetitle{#1}. \citeyear{#1}}

%% hypersetup
% Fit page automatically in PDF Viewer
\hypersetup{pdfstartview={Fit}}
\hypersetup{pdfinfo={
	Title=Causal Reasoning by Alexander Bochman,
	Author=Florian Endel,
	Subject=Seminar in AI
        %CreationDate={D:20241105170000},
        %ModDate={D:20241112170000},
	}
}


%=========================================================================================================================
\title{Causal Reasoning by Alexander Bochman}

\author[F. Endel]{Florian Endel}
\institute{Artificial Intelligence Group,\\
University of Hagen, Germany}
\date{December 10, 2024}
%=========================================================================================================================

%%%%%%%%%%%%%%%%%%%%%%%%%%%%%%%%%%%%%%%%%%%%%%%%%%%%%%%%%%%%%%%%%%%%%%

\newcommand{\hidelogo}{\logo{}}
\newcommand{\showlogo}
{
    \logo{\includegraphics[width=3cm]{figures/logoaig.png}}
}
\hidelogo

%=========================================================================================================================

\begin{document}

%=========================================================================================================================

{
    \setbeamertemplate{footline}{}
    \showlogo
    \begin{frame}
      \titlepage
    \end{frame}
}


\section{Introduction \& Background}

\begin{frame}{History: Overview of Bochman's Work on Causality}
    \note{
        \begin{itemize}
            \item historical context
            \item 2021: book - fundamental explanations, detailed introduction 
            \item no references before about 2000
            \item explored various aspects of causal reasoning over years (summarised in book)
            \item most recent work 2024 with a concise and fundamental approach 
            \item no co-authors 
        \end{itemize}
    }

    Relevant for my presentation: 
    \par
    \begin{itemize}
        \item \customciteNoAuthor{bochman_logical_2021}
        \item \textbf{\customciteNoAuthor{bochman_causal_2024}}
    \end{itemize}
    
    Early work: 
    \par
    \begin{itemize}        
        \item \customciteNoAuthor{bochman_logic_2003}
        \item \customciteNoAuthor{bochman_causal_2004}
    \end{itemize}
    
    Exploring various aspects: 
    \par
    \begin{itemize}    
        \item \customciteNoAuthor{bochman_actual_2018}
        \item \customciteNoAuthor{bochman_default_2023}
        \item \customciteNoAuthor{bochman_inferential_2023}
    \end{itemize}
    
\end{frame}


\begin{frame}{Relevance of causality} 
    \note{
        \begin{itemize}
            \item (impression of) causality is relevant in daily lives \& science
            \item [1] Kahneman: humans tend to interpret their observations in causal terms / cause and effect
            \item [2] Pearl: causal inference from observations $\neq$ experiments
            \begin{itemize}
                \item Structural Causal Model uses structural equation modeling
                \item Pearl "The Book of Why: The New Science of Cause and Effect" (2018), 'do-calculus, counterfactuals'
                \item \textbf{Causal inference} process of determining independent, actual effect of a particular phenomenon that is a component of a larger system. %The main difference between causal inference and inference of association is that causal inference analyzes the response of an effect variable when a cause of the effect variable is changed
                \item \textbf{Causal reasoning} is the process of identifying causality: the relationship between a cause and its effect. 
            \end{itemize}
            \item [3] show, that hints about causation (and derived recommendations) are common in literature; "causation" is not mentioned directly
        \end{itemize}
    }
    \begin{itemize}
        \item "The \highlight{interpretation of probability as propensity} leads people to base their judgments of likelihood primarily on \highlight{causal considerations}, and to ignore information that does not have causal significance."\footcite{kahneman_thinking_2013}
        \item "The possibility to \highlight{learn causal relationships from raw data} has been on philosophers' dream list since the time of Hume (1711-1776)" \footcite{pearl_causality_2000}
        \item "\highlight{causal intent}, inference, implications, and recommendations — \highlight{is common}" (in the observational  health literature) \footcite{haber_causal_2022}
    \end{itemize}
    \vspace{1em}
    ... but \yellowhighlight{how to formalize causality in formal logic}?
\end{frame}

\section{Logical Formal Theory of Causal Reasoning}

\begin{frame}{Bochman's approach} %[allowframebreaks]
    \note{
        \begin{itemize}
            \item concepts: causation, proposition, acceptance
            \item phil. terminology: rationality, normativity, reasons, explanations
            \item principles and constructions: inherently normative character 
            \item normative: expressing value judgments / not stating facts
            %\vspace{1em}
            %\item introduction references Hilberts formalisation of geometry
            %\item parable of the blind men and the elephant
            \item flexible exploration
            \vspace{1em}
            \item Semantics based on \highlight{causal principle of acceptance for propositions}
            \item Causal theory describes "why" certain propositions should be accepted through causal rules.
            \item Rational semantics only show which propositions should be accepted but don't explain the original "why."
            \item \textbf{Distinction}: cannot reconstruct the full causal origins of accepted propositions from their semantics alone
        \end{itemize}
    }
    \begin{itemize}
        %\item ... as a formal basis for a general theory of causal reasoning\footcite{bochman_causal_2024}
        \item Causation is multifaceted: no formalization of concepts 
        \item Approach: "investigate important variants of causal reasoning"\footcite{bochman_causal_2024}
        \item Definition of systems of reasoning: 
        \begin{itemize}
            \item \yellowhighlight{Language} consists of set of \highlight{(causal) inference rules} defined on a set of \highlight{propositions}
            \item \yellowhighlight{Semantics}: valuations of propositions (with causal rules)
            \item \yellowhighlight{Causal Inference}: formal derivations preserving \highlight{rational semantics}
        \end{itemize}
        \item Distinction: \highlight{rational semantics} vs. \highlight{causal theory}
        \item Inherently nonmonotonic
    \end{itemize}
\end{frame}

\begin{frame}{Causal Calculus} %[allowframebreaks]
    \note{
        \begin{itemize}
            \item 2021: two chapters about causal calculus: 
            \begin{itemize}
                \item "classic" Geffner 92, McCain and Turner 97, Lifschitz 97, Pearl 
                \item The causal calculus can be seen as a most natural and immediate generalization of classical logic that allows for causal reasoning.
            \end{itemize}
            \item Knowledge can be stored as cause - effect relationships
        \end{itemize}
    }
    \begin{itemize}
        \item ... general logical formalism of \yellowhighlight{causal reasoning}\footcite{bochman_actual_2018}
        \item 2 layers: nonmonotonic semantics + logics of causal rules \footcite{bochman_causal_2004}
    \end{itemize}

    \begin{block}{Causal Language}
        \begin{itemize}
            \item Built on top of classical logic
            \item Language: set of causal rules
            \item Underlying language $L$: set of proposition
        \end{itemize}
    \end{block}
    
    \begin{block}{Basics: classical propositional language \cite[79]{bochman_logical_2021}}
        \begin{itemize}
            \item $\land, \lor, \neg, \to, t, f$: classical connectives \& constants
            \item Th ($\vdash$): Syntactic \textit{provability}, based on formal proof rules.
            \item $\vDash$: Semantic \textit{entailment}, based on truth in all models.
            \item \textit{p,g,r} finite sets of propositions; \textit{A, B, C} classical propositions
        \end{itemize}
    \end{block}
\end{frame}
%\framebreak



\begin{frame}{Causal Theories and their Semantics}
    \note{
        \begin{itemize}
            \item causal theory \( \Delta \): arbitrary set of causal rules with constraints on \emph{acceptance} of propositions
            \item Causal Acceptance Principle: A proposition capital B is accepted with respect to a causal theory delta, iff a set of propositions (lowercase) a causes B; where all propositions in a are accepted
            \item Preservation Principle: inference rules "transmit" acceptance
            \item Leibniz’ Principle of Sufficient Reason: normative,  propositions require reasons for their acceptance
        \end{itemize}
    }
    \begin{block}{Causal rule}
        \begin{itemize}
            \item Causal binary relation / causal rule: \yellowhighlight{$a \Rightarrow B$}: \textit{a causes B}
            \item Causal theory \( \Delta \): arbitrary set of causal rules 
        \end{itemize}
    \end{block}
    
    \begin{block}{Principles of Acceptance}
        \begin{itemize}
            \item Causal Acceptance Principle: $B$ is \emph{accepted} \emph{iff} $a \Rightarrow B$, where all $A$ in $a$ are accepted
            \item \textbf{Preservation} Principle: If all propositions in $a$ are accepted, and $a$ causes $B$, then $B$ should be accepted.
            \item Principle of \textbf{Sufficient Reason}: Any proposition should have a cause for its acceptance.
        \end{itemize}
    \end{block}
\end{frame}



\begin{frame}{Rational Semantics}
    \note{
        \begin{itemize}
            \item  For any set $u$ of propositions and a causal theory $\Delta$: $\Delta(u)$ the set of all propositions B that are directly caused by $u$ in $\Delta$
            \item $\Delta(u)$ monotonic operator; $\subseteq$ = subset
            %\item viewing valuation v itself as a set of (accepted) propositions
            \item causal model: determines that a proposition is accepted in a model if and only if it has a cause in this model.
            \item Fixed Point: applying the operator $\Delta$ to u doesn't change the set: $\Delta(u)=u$
            \item any causal theory has at least one causal model 
            \item Causal Models: defined as a fixed point of the operator $\Delta(u)$.
            \item least model: iteratively applied, starting with empty set $\varnothing$
            \item Rational Semantics: full set of causal models (not just the least model) that satisfy $\Delta$. These models provide a range of valid interpretations of the theory.
        \end{itemize}
    }
    \begin{block}{Valuation on propositions for describing semantics}
        \begin{itemize}
            \item function $v \in \{0, 1\}^L$
            \item $v(A) = 1$: proposition $A$ is accepted (‘taken-true’) 
            \item $v(A) = 0$: \yellowhighlight{non-acceptance is not rejection of A}
            \item rejection: $v(\neg A) = 1$
            \item $\Delta(u) = \{ B \mid a \Rightarrow B \in \Delta, \, a \subseteq u \}$
        \end{itemize}
    \end{block}
    \begin{itemize}
        \item \yellowhighlight{Causal model}: fixed point of (accepted) propositions $v = \Delta(v)$
        \item Least model $\Delta()$: smallest model
        \begin{itemize}
            \item $u_0 = \emptyset, \quad u_1 = \Delta(u_0), \quad u_2 = \Delta(u_1), \,\dots$
        \end{itemize}        
        \item \yellowhighlight{Rational Semantics}: set of all causal models of a theory
    \end{itemize}
    
    
\end{frame}


\begin{frame}{Example}
    \note{
        \begin{itemize}
            \item if \emph{Rained} is accepted (with respect to such a causal theory)
            \item both \emph{Grasswet} and \emph{Streetwet} should be accepted
            \vspace{1em}
            \item any acceptable set of propositions that contains \emph{Grasswet} should either contain \emph{Rained} or \emph{Sprinkler} as its causes
            \vspace{1em}
            \item defications from causes to their effects and from effects to their possible causes are essential parts of causal reasoning
        \end{itemize}
    }

    \begin{exampleblock}{Causal theory}
        \[
        \text{Rained} \Rightarrow \text{Grasswet}
        \]
    
        \[
        \text{Sprinkler} \Rightarrow \text{Grasswet}
        \]
        
        \[
        \text{Rained} \Rightarrow \text{Streetwet}
        \]
    \end{exampleblock}

\end{frame}


\begin{frame}{Causal Inference}
    \note{
        \begin{itemize}
            \item "If a rule a causes A holds and a is extended to a larger set b, the rule still holds."
            \item "If a causes A and the combined set of a,A casuse B, then a casues B can be inferred by combining these rules."
            \item Reflexivity postulate: first postulate of Tarski consequence
            \item Reflexivity would make propositions self-justified (self-evident)
            \item one of the key differences between causal inference and deductive consequence
            \item  ‘omission’ creates the possibility of causal reasoning
            \item a causally implies B under the causal theory $\Delta$
            \item The definition of $\Rightarrow_\Delta$ is not about a single, specific causal rule, but rather about a generalized statement capturing all causal relationships derivable from the causal theory $\Delta$
        \end{itemize}
    }
    Formal derivations "\emph{metainferences}" among causal rules that always preserve the rational semantics:
    
    \begin{itemize}
        \item \textbf{Monotonicity:} $\text{If } a \Rightarrow A \text{ and } a \subseteq b, \text{ then } b \Rightarrow A. $
        \item \textbf{Cut:} $\quad \text{If } a \Rightarrow A \text{ and } a, A \Rightarrow B, \text{ then } a \Rightarrow B$.
    \end{itemize}
    
    \begin{alertblock}{Reflexivity and Causal assumptions} 
        \begin{itemize}
            \item $A \Rightarrow A$ does \textbf{not} hold by default
            \item $A \Rightarrow A$ is a self-evident proposition that does not require further justification for its acceptance
        \end{itemize}
    \end{alertblock}    
    \begin{itemize}
        \item \textbf{$\Rightarrow_\Delta$}: all causal rules that are derivable from $\Delta$; $a \Rightarrow_\Delta B$
        \item Any causal theory $\Delta$ is semantically equivalent to $\Rightarrow_\Delta$
        
    \end{itemize}
    
    
\end{frame}



\begin{frame}[allowframebreaks]{Causal Operator $\mathcal{C}$ }
    \begin{itemize}
        \item $\mathcal{C}(u)$ is the set of propositions ($A$) that are caused by a set $u$:  $\mathcal{C}(u) = \{ A \mid u \Rightarrow A \}$
        %\item Every proposition $A$ in $\mathcal{C}(u)$ is a causal (and not "only" logical) consequence of $u$. 
        \item Similar to the derivability operator $Th(u)$ %: set of all propositions that can be formally derived from $u$ (i.e., syntactically provable propositions)
        \item \yellowhighlight{\textit{Monotonicity}}: adding more propositions to the set $u$, the set of consequences $\mathcal{C}(u)$ only grows or stays the same – it cannot shrink.
        \begin{itemize}
            \item $\text{If } u \subseteq v, \text{ then } \mathcal{C}(u) \subseteq \mathcal{C}(v)$
        \end{itemize}
        \item \yellowhighlight{\textit{Deductive Closure}}: $\mathcal{C}(u)$ always results in a \emph{deductively closed set}: all logical consequences of the elements in $\mathcal{C}(u)$ are also included in $\mathcal{C}(u)$
        \begin{itemize}
            \item $\mathcal{C}(u) = Th(\mathcal{C}(u))$
            \item e.g.: if the rain ($u$) causes the street to be wet ($\mathcal{C}(u)$), it also causes the logical consequences $Th(\mathcal{C}(u))$ (e.g., a slippery street)
        \end{itemize}
        \item \yellowhighlight{Transitivity}: if $A \Rightarrow B$ and $B \Rightarrow C$, then $A \Rightarrow C$
    \end{itemize}

\framebreak

    \begin{block}{Non-inclusivity of $\mathcal{C}$}
        $u$ is not necessarily a subset of $\mathcal{C}(u)$: $u \nsubseteq \mathcal{C}(u)$
    \end{block}
    \begin{itemize}
        \item set of propositions that are \highlight{caused} by $u$ (i.e., $\mathcal{C}(u)$) do not always contain $u$ itself
        \item just because a proposition is in $u$ it must \highlight{not} appear in the set of consequences caused by $u$
        \item Non-inclusivity related to (non-) reflexivity
        \item Contrast to derivability operator $\text{Th}$: original set is always included in the set of derivable propositions
    \end{itemize}
    \begin{exampleblock}{Example: $\mathcal{C}(\text{rain})$}
        a specific raining event ($u$) has a causal consequences ($\mathcal{C}(u)$, e.g. a wet street), but does not necessarily cause itself (but perhaps other raining events $u'$)
    \end{exampleblock}
    
\framebreak

    \begin{block}{Non-idempotence of $\mathcal{C}$}
        \begin{itemize}
            \item $\mathcal{C}(\mathcal{C}(u)) \neq \mathcal{C}(u)$    
            \item a set of propositions that are caused by $u$ ($\mathcal{C}(u)$) might cause another set of propositions $\mathcal{C}(\mathcal{C}(u))$
            \item causation propagates across several steps (cascading effects)
        \end{itemize}
    \end{block}
    \begin{exampleblock}{Example: $u = \{\text{rain}\}$}
        \begin{itemize}
            \item 1. Step: Direct causal effects of $u$: $\mathcal{C}(u)=\{\text{wet street}\}$
            \begin{itemize}
                \item causes logical consequences $Th(\mathcal{C}(u))$ directly: slippery street
            \end{itemize}
            \item 2. Step: Effects of $\mathcal{C}(u)=\{\text{slippery street}\}$: 
            \begin{itemize}
                \item $\mathcal{C}(\mathcal{C}(u)) = \{\text{people slip on the street}, \text{traffic slows down}\}$
            \end{itemize}
            \item 3. Step: Effects of $\mathcal{C}(\mathcal{C}(u))$: 
            \begin{itemize}
                \item $\mathcal{C}(\mathcal{C}(\mathcal{C}(u))) = \{\text{people get hurt}, \text{cloth get wet}\}$
            \end{itemize}
        \end{itemize}
    \end{exampleblock}

\end{frame}

\begin{frame}{Causal Theories and Expanded Notions}

    \begin{itemize}
        \item A \textbf{causal theory} is a set of causal rules (conditionals) that define the causal behavior within a certain framework.
        
        \item $ \mathcal{C}_{\Delta} $: For any causal theory $ \Delta $, there is a least production relation, that includes all causal rules derivable from $ \Delta $.

    \end{itemize}

\bigskip
    \begin{exampleblock}{}
    If theory $ \Delta $ states:
    $$
    \{A \Rightarrow B, B \Rightarrow C\},
    $$
    then for a premise set $ u = \{A\} $, one can systematically derive
    $$
    \mathcal{C}_{\Delta}(u) = \{B, C\},
    $$
    representing both intermediate and direct effects triggered by $ A $.
        
    \end{exampleblock}
\end{frame}


\begin{frame}{Causal Inference (\( \Rightarrow \)) vs. Deductive Consequence (\( \vdash_\Rightarrow \))}
    \note{
        \begin{itemize}
            \item \textbf{Causal Inference (\( \Rightarrow \))}: Refers to reasoning based on \emph{cause-and-effect} relationships defined by a causal theory.            
            \item \textbf{Deductive logic}: the propositional theory tells everything about the consequence relation (rules + results).
            \item Causal inference relation $\Rightarrow$ has the same propositional theories as the corresponding consequence relation \( \vdash_\Rightarrow \)
        \end{itemize}
    }

    \begin{itemize}
        \item \textbf{\( \Rightarrow \)}: reasoning based on \emph{cause-and-effect} relationships
        \item \textbf{\( \vdash_\Rightarrow \)}: propositional theory about the consequence relation (rules + results).
        \item $\Rightarrow$ has the same propositional theories as \( \vdash_\Rightarrow \)
        \item \textbf{But}: causal reasoning retains extra causal information beyond what is encoded in the propositional theories.
    \end{itemize}

    \begin{exampleblock}{Example}
        \begin{itemize}
            \item $Rained \Rightarrow Grasswet$, $Sprinkler \Rightarrow Grasswet$ lead to $Grasswet$
            \item \textbf{But} the \yellowhighlight{specific causal mechanisms} (rain vs. sprinkler) are \yellowhighlight{part of the causal model}, not captured in the resulting propositional theory.
        \end{itemize}            
    \end{exampleblock}
\end{frame}

\begin{frame}{Equivalence}
    \note{
        \begin{itemize}
            \item  Two causal theories are logically equivalent, if each can be obtained from the other using the postulates of causal inference
            \item Reverse: rational semantics does not fully determine the content of the original causal theory.
            \item develops a notion of \emph{strong semantic qeuivalency}
            \item Example: single model $\emptyset$, no props. accepted        
        \end{itemize}
    }

    \begin{itemize} 
        \item Logically equivalent $\Delta$ are semantically equivalent
        \item Reverse does not hold!
    \end{itemize}
    \begin{exampleblock}{}
        \begin{itemize}
            \item Two theories: $$\Delta = \{A \Rightarrow B\}, \Phi = \{A \Rightarrow C\}$$
            \item different, determine same rational semantics $\Delta() = \Phi() = \emptyset$
            \item add causal rule $A \Rightarrow A$ to both 
            \item $\Delta() = \{A, B\}, \quad \Phi()=\{A,C\}$
        \end{itemize}
    \end{exampleblock}
\end{frame}


\begin{frame}{Axioms vs. Assumptions}
    \note{
        \begin{itemize}
            \item Law of causality (Leibniz): any accepted proposition should have an accepted cause
            \item A proposition \( A \) will be called an \textbf{axiom} of a causal theory \( \Delta \) if the rule \( \emptyset \Rightarrow A \) belongs to \( \Delta \);
            \item axioms do not require justification, must be accepted in any model
            \item A proposition \( A \) will be called a \textbf{causal assumption} of a causal theory if the rule \( A \Rightarrow A \) belongs to it.  
            \item self-evident propositions, can be incorporated into a causal model when cosistent
            \item any axiom will also be an assumption, though not vice versa
            \item As a result, causal theories admit in general multiple causal models depending on the assumptions we actually accept
            \item abductive reasoning: Bochman 2007
        \end{itemize}
    }

    \begin{itemize} 
        \item \textbf{Axiom}: \( \emptyset \Rightarrow A \) 
        \item \textbf{Causal Assumption}: \( A \Rightarrow A \) 
        \item (Relationship to abductive reasoning \cite{bochman_logical_2021})
\end{itemize}
    \begin{exampleblock}{Original Example: Rained, Sprinkler, Streetwet}
        \begin{itemize}
            \item  by itself, this causal theory has a single empty causal model $\emptyset$
            \item add assumptions: $Rained \Rightarrow Rained \quad Sprinkler \Rightarrow Sprinkler$
            \item Rational semantics of this causal theory: three causal models
            \item $\{Rained, Grasswet, Streetwet\}$, $\{Sprinkler, Grasswet\}$, $\{Rained, Sprinkler, Grasswet, Streetwet\}$
        \end{itemize}
    \end{exampleblock}
\end{frame}


\begin{frame}{Supraclassical Causal Reasoning}
    \note{
        \begin{itemize}
            \item  ancient principle ex nihilo nihil fit (‘Nothing comes from nothing”)
            \item  ex falso quodlibet (“from falsehood, anything”)
        \end{itemize}
    }

    \begin{itemize} 
        \item Aim: integrating causal reasoning into a more comprehensive reasoning system
        \item \textbf{Classical entailment} (formal logical reasoning): integral part of the causal system
        \item Supraclassical: extends and includes classical entailment, to naturally incorporate classical logic
        %\item But: no way of expressing the usual truth-tables
        \item "Causal reasoning is \highlight{not a replacement or competitor} of logical (deductive) reasoning, but its complement (or extension)" \cite{bochman_logical_2021}
    \end{itemize}
    \begin{itemize}
        \item \textit{(Strengthening)} If $A \mathhighlight{\vDash} B$ and $B \Rightarrow C$, then $A \Rightarrow C$;
        \item \textit{(Weakening)} If $A \Rightarrow B$ and $B \mathhighlight{\vDash} C$, then $A \Rightarrow C$;
        \item \textit{(And)} If $A \Rightarrow B$ and $A \Rightarrow C$, then $A \Rightarrow (B \land C)$;
        \item \textit{(Truth and Falsity)} $t \Rightarrow t$; $\darkmathhighlight{f \Rightarrow f}$.
    \end{itemize}

\end{frame}


\begin{frame}{Further Properties of Causal Reasoning}
    \note{
        \begin{itemize}
            \item Defaults in Causal Reasoning: special kind of assumptions, must accept unless there is reason not to (not rejected $\neg A$)
            \item nonmonotonic formalisms, adding auxiliary ‘presumptions’ to an inference rule such that only their refutation could lead to cancellation of the rule
            \item SEM: cause-and-effect recipes 
            \item new truths into the causal framework can invalidate earlier conclusions about causation.
            \item When \textbf{negation} is viewed as default, negative propositions are exempted from the \textbf{need of causal explanation}; in other words, they do not need causes for their acceptance. 
        \end{itemize}
    }

    \begin{itemize} 
        \item \textbf{Default causal theory}: pair $(\Delta, \mathcal{D})$, $\mathcal{D}$ a subset of causal assumptions
        \item \textbf{Nonmonotonicity}
        \begin{itemize}
            \item Context-Sensitive Causal Reasoning
            \item Causal acceptance is directional
        \end{itemize}
        \item \textbf{Structural Equation Models}: representation in the causal calculus
        \item Counterfactual Equivalence: related to SEM
        \item Negative Causal Completion: negation as default ($ \neg p \Rightarrow \neg p$)
    \end{itemize}

\end{frame}


%%%%%%%%%%%%%%%%%%%%%%%%%%%%%%55




\section{Closing remarks}
\begin{frame}{Applications of Causal Reasoning}
    \begin{itemize}
        \item Explainable AI \footcite{bochman_causal_2024}
        \item Causal attribution (actual causality) in legal theory \footcite{bochman_actual_2018}
    \end{itemize}
\end{frame}


\begin{frame}{Take away messages}
    \note{
        \begin{itemize}
            \item It serves as a natural basis for integrating Pearl’s approach to causation while also accommodating classical entailment within causal reasoning.
            \item A critical feature is the \textbf{asymmetry}: while the language determines its semantics, the reverse (semantics determining the language) is not true. This challenges conventional representational approaches.
            \item ties back to historical views on causation from Aristotle, Leibniz, and Hume
            \item It formalizes meaning through \textbf{rules of inference} (causal relationships). It provides a philosophical basis for understanding asymmetrical relationships in causal reasoning systems.
            \item \textbf{holistic approach} in his theory, where propositions make sense primarily in relation to the broader web of inferences rather than in isolation.
        \end{itemize}
    }
    Bochman's causal reasoning...
    \begin{itemize}
        \item Unified approach: integrate Pearl's approach to causation, classical entailment within causal reasoning
        \item Nonmonotonic reasoning
        \item Asymmetry in Semantics
        \item Roots in Historical Reasoning
        \item Connection to Inferentialism (rule-based vs. representational)
    \end{itemize}
\end{frame}



%%-------------------------------------------------------------------------------------------------------------

%\logo{\includegraphics[width=3cm]{figures/logoaig.png}}
{
    \setbeamertemplate{footline}{}
    \showlogo
    \begin{frame}
      \titlepage
    \end{frame}
}
\nologo


%%-------------------------------------------------------------------------------------------------------------

\appendix
\section{Appendix}
\subsection{Bibliography}

\begin{frame}[allowframebreaks]
    \frametitle{References}
    {\footnotesize
        \printbibliography
    }
\end{frame}


\section{Spare Slides}

\begin{frame}{History: Other relevant authors and papers}
    Some of the most influential authors:
    \vspace{1em}
    \begin{itemize}
        \item \customcite{pearl_embracing_1988}
        \item \customcite{pearl_causality_2000}
        \item \customcite{lifschitz_logic_1997}
    \end{itemize}
    %\framebreak 
    \begin{itemize}
        \item \customcite{geffner_causal_1990}
        \item \customcite{turner_logic_1999}
        \item \customcite{dash_causal_2012}
    \end{itemize}
    \vspace{1em}
    ... and many others
\end{frame}



\subsection{Classical propositional language}

% 1: Syntactic Provability
\begin{frame}{Syntactic Provability (\( \vdash \))}
    \textbf{Definition:}  
    \( Th \) (\( \vdash \)) represents \highlight{syntactic provability}, meaning a proposition $A$ can be formally derived using axioms and formal inference rules.  
    Provability is about whether a proposition $A$ can be formally derived from a set of axioms or premises \( \Gamma \) using a given set of syntactic proof rules within a formal system.

    \textbf{Key Idea:} 
    \begin{itemize}
        \item A proposition $A$ is provable (written as $Th \vdash A$ if there is a formal proof sequence for $A$.
        \item Relies entirely on the structure and rules of a formal system (e.g., axioms, inference rules like Modus Ponens).
        \item Does not involve interpretations or models; it’s purely rule-based reasoning.
        \item Following strict rules to deduce a conclusion.

    \end{itemize}                 

\end{frame}

% 2: Semantic Entailment
\begin{frame}{Semantic Entailment (\( \vDash \))}
    \textbf{Definition:}  
    \( \vDash \) represents \highlight{semantic entailment}, meaning a proposition $A$ is true in any model where the assumptions \( \Gamma \) hold.  Entailment is about whether $A$ is true in every possible model (interpretation) where a set of premises Γ is true.

    
    \textbf{Key Idea:} 
    \begin{itemize}
        \item Determines logical consequence based on truth in all models, not formal derivation.
        \item Like checking universal truths: something is true in all possible worlds where the premises hold.
        \item Depends on truth values assigned to propositions in all models of the logic.
        Assesses whether a logical proposition holds in a model-theoretic sense.    
    \end{itemize}  

\end{frame}

% 3: Propositional Atoms
\begin{frame}{Propositional Atoms (\( p, g, r \))}
    \textbf{Definition:}  
    Propositional atoms are the smallest, indivisible units of logic, representing atomic facts.  

    \textbf{Example:}
    \[
        p = \text{"It is raining"}, \quad g = \text{"The grass is wet"}, \quad r = \text{"It is windy"}.
    \]
    No further breakdown is possible for these logical units.
\end{frame}

% 4: Classical Propositions
\begin{frame}{Classical Propositions (\( A, B, C \))}
    \textbf{Definition:}  
    Classical propositions are statements formed by combining propositional atoms with logical connectives (\( \land, \lor, \lnot, \rightarrow \)) or truth constants (\( t, f \)).  

    \textbf{Example:}
    \[
        A = p \land g, \quad B = p \lor \lnot r, \quad C = \lnot g \rightarrow r.
    \]
    These statements are more complex and their truth values depend on logical interpretation.
\end{frame}

% Summary
\begin{frame}{Summary}
    \begin{itemize}
        \item Syntactic Provability (\( \vdash \)): Formal derivation based on rules.  
        \item Semantic Entailment (\( \vDash \)): Truth across all models.
        \item Propositional Atoms: Indivisible units (\( p, g, r \)).  
        \item Classical Propositions: Composite logical statements (\( A, B, C \)).  
    \end{itemize}

    \vspace{0.5cm}
    These concepts bridge syntactic and semantic aspects of logic, fundamental to understanding relationships in formal reasoning systems.
\end{frame}

% Examples
\begin{frame}{Examples}
    \begin{itemize}
        \item \textbf{Provability}: $ Th \vdash Grasswet $ \par
        Derive \textit{Grasswet} syntactically using axioms like $ Rained \rightarrow Grasswet $.
        
        \item \textbf{Entailment}:  $ Rained \vDash Grasswet $ \par
        \textit{Grasswet} is true in every model where \( Rained \) is true.
        
        \item \textbf{Propositional Atoms}:
        \( Rained \), \( Grasswet \), and \( Streetwet \) are indivisible facts used to construct more complex statements.
        
        \item \textbf{Classical Proposition}: $ Rained \rightarrow (Streetwet \lor Grasswet) $ \par
        A composite logical statement describing causal or correlative relationships.
    \end{itemize}
\end{frame}



\section{Example slides}

\begin{frame}{Highlighting Text}
    We can \highlight{highlight} text. There are also a \darkhighlight{darker} option, and a \yellowhighlight{yellow} option.\\
    The same options are also available in math mode: $\mathhighlight{a} + \darkmathhighlight{b} = \yellowmathhighlight{c}$
\end{frame}

\begin{frame}{Blocks}
    \begin{block}{Block}
        This is a regular block.
        \begin{itemize}
            \item This is an item in a block.
        \end{itemize}
    \end{block}
    
    \begin{alertblock}{Block}
        This is an alert block.
        \begin{itemize}
            \item This is an item in an alert block.
        \end{itemize}
    \end{alertblock}
    
    \begin{exampleblock}{Block}
        This is an example block.
        \begin{itemize}
            \item This is an item in an example block.
        \end{itemize}
    \end{exampleblock}
\end{frame}


\end{document}
